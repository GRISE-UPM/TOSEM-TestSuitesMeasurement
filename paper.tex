%%
%% This is file `sample-acmsmall.tex',
%% generated with the docstrip utility.
%%
%% The original source files were:
%%
%% samples.dtx  (with options: `acmsmall')
%% 
%% IMPORTANT NOTICE:
%% 
%% For the copyright see the source file.
%% 
%% Any modified versions of this file must be renamed
%% with new filenames distinct from sample-acmsmall.tex.
%% 
%% For distribution of the original source see the terms
%% for copying and modification in the file samples.dtx.
%% 
%% This generated file may be distributed as long as the
%% original source files, as listed above, are part of the
%% same distribution. (The sources need not necessarily be
%% in the same archive or directory.)
%%
%% The first command in your LaTeX source must be the \documentclass command.
\documentclass[acmsmall]{acmart}

% Packages
\usepackage{multirow}
\usepackage{adjustbox}
\usepackage{framed}
\usepackage{Sweave}
\usepackage{balance} % For balanced columns on the last page
\usepackage{listings}
\usepackage{cprotect}
\usepackage{hyperref}

% Definitions for listings
\definecolor{dkgreen}{rgb}{0,0.6,0}
\definecolor{gray}{rgb}{0.5,0.5,0.5}
\definecolor{mauve}{rgb}{0.58,0,0.82}
\lstset{frame=none, % tblr
  language=Java,
  escapeinside={\%*}{*)},          % if you want to add LaTeX within your code
  aboveskip=3mm,
  belowskip=3mm,
  showstringspaces=false,
  columns=flexible,
  basicstyle={\footnotesize\ttfamily},
  numbers=none,
  numberstyle=\tiny\color{gray},
  keywordstyle=\color{blue},
  commentstyle=\color{dkgreen},
  stringstyle=\color{mauve},
  breaklines=true,
  breakatwhitespace=true,
  tabsize=3,
  moredelim=[is][\sout]{¡}{¡},
}

% Oscar's notes
\newcommand{\odnote}[1]{\textcolor{blue}{[OD: #1]}}

% Graphics path
\graphicspath{{./figures/}}

%%
%% \BibTeX command to typeset BibTeX logo in the docs
\AtBeginDocument{%
  \providecommand\BibTeX{{%
    \normalfont B\kern-0.5em{\scshape i\kern-0.25em b}\kern-0.8em\TeX}}}

%% Rights management information.  This information is sent to you
%% when you complete the rights form.  These commands have SAMPLE
%% values in them; it is your responsibility as an author to replace
%% the commands and values with those provided to you when you
%% complete the rights form.
\setcopyright{acmcopyright}
\copyrightyear{2021}
\acmYear{2021}
\acmDOI{10.1145/9999999.9999999}


%%
%% These commands are for a JOURNAL article.
\acmJournal{TOSEM}
\acmVolume{99}
\acmNumber{9}
\acmArticle{999}
\acmMonth{9}

%%
%% Submission ID.
%% Use this when submitting an article to a sponsored event. You'll
%% receive a unique submission ID from the organizers
%% of the event, and this ID should be used as the parameter to this command.
%%\acmSubmissionID{123-A56-BU3}

%%
%% The majority of ACM publications use numbered citations and
%% references.  The command \citestyle{authoryear} switches to the
%% "author year" style.
%%
%% If you are preparing content for an event
%% sponsored by ACM SIGGRAPH, you must use the "author year" style of
%% citations and references.
%% Uncommenting
%% the next command will enable that style.
%%\citestyle{acmauthoryear}

%%
%% end of the preamble, start of the body of the document source.
\begin{document}

%%
%% The "title" command has an optional parameter,
%% allowing the author to define a "short title" to be used in page headers.
\title{Test cases as a measurement instrument in experimentation}

%%
%% The "author" command and its associated commands are used to define
%% the authors and their affiliations.
%% Of note is the shared affiliation of the first two authors, and the
%% "authornote" and "authornotemark" commands
%% used to denote shared contribution to the research.
\author{O. Dieste}
\email{odieste@fi.upm.es}
\orcid{0000-0002-3060-7853}
\affiliation{%
  \institution{Universidad Polit\'ecnica de Madrid}
  \city{Boadilla del Monte}
  \postcode{28660}
  \country{Spain}
}

\author{F. Uyaguari}
\email{fuyaguar@etapa.net.ec}
\orcid{0000-0001-7060-1002 }
\affiliation{%
  \institution{ETAPA Telecommunications Company}
  \city{Cuenca}
  \postcode{10204}
  \country{Ecuador}
}

\author{S. Vegas}
\email{svegas@fi.upm.es}
\orcid{0000-0001-8535-9386}
\affiliation{%
  \institution{Universidad Polit\'ecnica de Madrid}
  \city{Boadilla del Monte}
  \postcode{28660}
  \country{Spain}
}

\author{I. Panach}
\email{J.Ignacio.Panach@uv.es}
\orcid{0000-0002-7043-6227}
\affiliation{%
  \institution{University of Valencia}
  \city{Valencia}
  \postcode{46010}
  \country{Spain}
}

\author{N. Juristo}
\email{natalia@fi.upm.es}
\orcid{0000-0002-2465-7141}
\affiliation{%
  \institution{Universidad Polit\'ecnica de Madrid}
  \city{Boadilla del Monte}
  \postcode{28660}
  \country{Spain}
}

%%
%% By default, the full list of authors will be used in the page
%% headers. Often, this list is too long, and will overlap
%% other information printed in the page headers. This command allows
%% the author to define a more concise list
%% of authors' names for this purpose.
\renewcommand{\shortauthors}{O. Dieste, et al.}

%%
%% The abstract is a short summary of the work to be presented in the
%% article.
\begin{abstract}
\textbf{Background:} Test suites are frequently used to quantify relevant software attributes, such as quality or productivity. \textbf{Problem:} We have detected that the same response variable, measured using different test suites, yields different experiment results. \textbf{Aims:} Assess to which extent differences in test case construction influence measurement accuracy and experimental outcomes. \textbf{Method:} Two industry experiments have been measured using two different test suites, one generated using an \textit{ad-hoc} method and another using \textit{equivalence partitioning}. The accuracy of the measures has been studied using standard procedures, such as ISO 5725, Bland-Altman and Interclass Correlation Coefficients. \textbf{Results:} There are differences in the values of the response variables up to $\pm 60\%$, depending on the test suite (\textit{ad-hoc} vs. \textit{equivalence partitioning}) used. \textbf{Conclusions:} The disclosure of datasets and analysis code is insufficient to ensure the reproducibility of SE experiments. Experimenters should disclose all experimental materials needed to perform independent measurement and re-analysis.
\end{abstract}

%%
%% The code below is generated by the tool at http://dl.acm.org/ccs.cfm.
%% Please copy and paste the code instead of the example below.
%%
\begin{CCSXML}
<ccs2012>
<concept>
<concept_id>10002944.10011123.10010916</concept_id>
<concept_desc>General and reference~Measurement</concept_desc>
<concept_significance>500</concept_significance>
</concept>
<concept>
<concept_id>10002944.10011123.10011131</concept_id>
<concept_desc>General and reference~Experimentation</concept_desc>
<concept_significance>500</concept_significance>
</concept>
</ccs2012>
\end{CCSXML}

\ccsdesc[500]{General and reference~Measurement}
\ccsdesc[500]{General and reference~Experimentation}

%%
%% Keywords. The author(s) should pick words that accurately describe
%% the work being presented. Separate the keywords with commas.
\keywords{Test suite, measuring instrument, accuracy, agreement}

%% A "teaser" image appears between the author and affiliation
%% information and the body of the document, and typically spans the
%% page.
%\begin{teaserfigure}
%  \includegraphics[width=\textwidth]{sampleteaser}
%  \caption{Seattle Mariners at Spring Training, 2010.}
%  \Description{Enjoying the baseball game from the third-base
%  seats. Ichiro Suzuki preparing to bat.}
%  \label{fig:teaser}
%\end{teaserfigure}
%
%%%
%% This command processes the author and affiliation and title
%% information and builds the first part of the formatted document.
\maketitle

\section{Introduction}\label{sec:introduction}

Test-driven development (TDD) research frequently uses the external quality (QLTY) and productivity (PROD) response variables. QLTY is typically measured as the ''amount'' of correct functionality delivered by the developers' code. PROD has a similar definition but is related to a time frame (e.g., the duration of an experimental session). "Functionality" is an abstract concept, not directly observable. In TDD research, test cases are often used as surrogates of functionality.

We have conducted a family of experiments on TDD, as part of the Empirical Software Engineering Industry Lab (ESEIL) project. We used different test suites, as recommended by Shadish et al. \cite[81-82]{Shadish2002}, to measure QLTY and PROD values, thus preventing the mono-method threat to validity. We anticipated some variability among measures, but differences were much larger than we expected. The experimental analyses yield different results, sometimes reversing the effect of the independent variables, depending on the test suite used \cite{Elizabeth2015}.

This paper aims at evaluating to what extent test cases influence the measurement of response variables in TDD experiments. Although the discussion is specifically framed in TDD research, measurement using test cases is frequent in software engineering (SE) research, e.g., \cite{kieburtz1996software,knight1986experimental,feldt1998generating}; other SE areas can thus benefit from our findings. 

The contributions of this paper are:
\begin{itemize}

\item We show that the results of TDD experiments vary depending on the test suites used as measuring instruments. We have assessed this fact in our experiments, but we are certain that the same harmful effect happens in other TDD experiments.

\item We introduce specialized terminology and methods, borrowed from Metrology, the Natural and the Social sciences, to study the accuracy of the test suites when used as measuring instruments.

\item We assess that the measures made using different test suites yield very different results. The same piece of code may exhibit $\pm 60\%$ score differences depending on the test suite used.

\item The publication of datasets and analysis code, as currently required by some publishers, may be sufficient for ensuring reproducibility \cite{NAP25303,fernandez2019open}, but insufficient to evaluate the influence of the measuring instruments. We propose some recommendations to improve the situation: (1) experiments should disclose all experimental materials needed to perform independent measurements, and (2) the practice of re-analysis \cite{mittelstaedt1984econometric,IJzendoorn1994} should be adopted in SE to improve experimental research. 

\end{itemize}

This paper has been written using reproducible research principles. The manuscript \LaTeX~code  is available at \url{https://github.com/GRISE-UPM/TestSuitesMeasurement} (including data files, Java and R code). Analyses have been carried out using R \cite{R} version 4.0.2 (2020-06-22), and the packages \textit{lme4} \cite{lme4}, \textit{xtable} \cite{xtable}, \textit{texreg} \cite{texreg}, \textit{broom} \cite{broom}, \textit{MethComp} \cite{MethComp}, \textit{Hmisc} \cite{Hmisc}, and  \textit{emmeans} \cite{emmeans}.

The paper is structured as follows: Section~\ref{sec:problem-description} describe the research problem. Section~\ref{sec:objectives} sets out the research goals. In Section~\ref{sec:comparison} we introduce the terminology and methods used in Metrology and other sciences for the comparison of measuring instruments. The actual comparison is performed in Section~\ref{sec:comparison-results}. We discuss the implication of our findings in Section~\ref{sec:discussion} and, finally, provide some recommendations in Section~\ref{sec:conclusions}.
\input {Rsections/problem}
% !TEX encoding = UTF-8 Unicode
\section{Research questions and methodology}\label{sec:objectives}

\subsection{Research questions}\label{sec:questions}

Test suites are being used routinely as measuring instruments in TDD experiments, e.g., \cite{Causevic2012,Desai2009,Erdogmus2005,Fucci2013}. TDD experiments are being combined through meta-analysis \cite{rafique2012effects}. We have shown that the experimental results are conditional on the test suites. The same applies, indirectly, to the meta-analyses based on those TDD studies. 

We are concerned about the use of test suites as measuring instruments. We aim to evaluate to which degree similar test suites, e.g., with comparable branch coverage, give different measures. \textbf{A better understanding of the role of test suites for measurement will provide decision criteria for the selection or construction, utilization, and sharing of test suites in SE experiments}.

To the best of our knowledge, \textbf{this problem has not been addressed in the SE literature}. Given its relevance for the TDD community (and from a general perspective to the entire empirical SE), this paper sets out the following research questions:

\vspace{0.8mm}

\textbf{RQ1:} \textit{How} can we assess the accuracy of the measures obtained using test suites?

\vspace{0.8mm}

Measurement is a complex process. Scientists and engineers have developed specific procedures to assess the accuracy of measures, and compare measurement instruments. These procedures can be applied to test suites.

\vspace{0.8mm}

\textbf{RQ2:} \textit{How much} do the AH and EP datasets differ from each other?

\vspace{0.8mm}

The statistical analyses in Section~\ref{sec:problem-description} yield clearly different results. However, such results do not provide an indication of the extent to which the AH and PE datasets differ from each other. Common sense suggests that the differences are large, but we miss a concrete description of \textit{how large} they are.

\subsection{Research method}\label{sec:method}

The research questions posed above imply the comparison of two sets of measurements (AH and EP) generated using different test suites (\textit{ad-hoc} and \textit{equivalence partitioning}). 

The comparison of measurements is not new in SE. Quite a few papers address the comparison of metrics, e.g., \cite{basili1981evaluating,zhang2007performance,zhao1998comparison,jiang2008comparing,di2007comparing}. However, these works do not put the metrics themselves into question, but they typically examine their predictive ability to choose the ''best'' metric for a purpose. Other works, e.g., \cite{meneely2012validating} provide metric validation criteria, but these criteria do not include procedures and methods to compare metrics and decide which ones are more accurate. To conclude, \textbf{we miss theoretical foundations to analyze and compare measurements in SE}.

In turn, different scientific disciplines (e.g., Medicine, Psychology, and Metrology particularly) have dealt with the problem of comparing measurements, giving rise to different comparison approaches. To the best of our knowledge, none of them has been used in SE so far.

\textbf{To answer RQ1}, we provide in Section~\ref{sec:comparison} an abridged description of the different comparison approaches that apply to our research problem.

\textbf{To answer RQ2}, we apply in Section~\ref{sec:comparison-results} all suitable comparison procedures to the AH and EP datasets, with a threefold purpose: (1) quantify how large the differences between measurements are, (2) illustrate how the different comparison approaches can be used in practice, and (3) choose the simplest procedure for routinely use in SE.
\input {Rsections/comparison}
\input {Rsections/results}
\input {Rsections/discussion}
\section{Conclusions}\label{sec:conclusions}

The software metrics area is quite mature in SE. Their formal properties \cite{fenton2014software} and some common pitfalls, e.g., \cite{fenton1992software}, are well understood. However, when dealing with experimental data, it seems that we have overlooked, at least partially, the complexities of measurement. There are several reasons for that: (1) in many cases, the metrics and measuring instruments should be specifically designed for an experiment, (2) the entities of interest in SE are often theoretical constructs, so that objective measurement instruments cannot, by definition, be ever available, (3) we probably trust in excess in the power of statistics, etc.

We have confirmed in this research that different test suites (often used as measuring instruments) give different measures on the same program. Such differences are so radical that they reverse the effects of the factors in the statistical analyses. We have restricted our inquiry to the response variable \textit{external quality}, frequently used in TDD experiments. However, we believe that our findings can be extrapolated to other response variables, research areas and even other research methods, e.g., case studies.

Experimentation is not mature enough to introduce standard measures and measurement instruments. Of course, benchmarks can be adopted. However, the mere adoption of a benchmark does not solve the problems described in this paper because nothing guarantees that such a benchmark provides the \textit{right} measures. Even so, some action should be taken to avoid the harmful effects of metrics and measuring instruments in SE experimental research. In our opinion, three measures can be beneficial for the experimental community:

\begin{itemize}

\item Researchers should disclose not only the measurement results, i.e., the refined data which proceeds to analysis but also the \textbf{raw data} (e.g., subjects' code) and the \textbf{measurement procedure and instruments}. This enables later critical examination and the conduction of a wide range of replications, in particular, re-analysis \cite{mittelstaedt1984econometric,IJzendoorn1994}, where the raw data is independently re-processed before analysis. 

\item The properties of the measures and measuring instruments should be considered before actual measurement takes place. In many cases, measurement standards, e.g., programs satisfying subsets of requirements, can be easily created well before the experiment is conducted, so that formal analysis and empirical studies are possible.

\item When standards are not available, experimenters could use different measures and instruments to avoid threats to construct validity. Coherent results obtained with different instruments would increase the confidence in the experiment results. Just to cite an example, in the PT and EC experiments, the \textit{Treatment} (ITLD vs. TDD) \textbf{was largely unaffected} by the AH and EP test suites. This fact provides us some relief. If the treatments were unaffected, the meta-analyses and other secondary studies based on these data would produce correct results.

\end{itemize}

Finally, in this research, we have only addressed the influence of the measuring instruments in the measurement results. However, the measurement process, as indicated in Section~\ref{sec:comparison}, has several components. They all (in particular, the measurer and the manipulations before applying the measuring instrument) can influence the measurement result too. The assessment of the impact of those elements will be future research.

\section{Acknowledgments}

This work was partially supported by the Spanish Ministry of Economy and Competitiveness research grant TIN2014-60490-P, and the Finish TEKES research grant ESEIL (FiDiPro scholarship).

%% Balance the columns in the last page
\balance

%%
%% The next two lines define the bibliography style to be used, and
%% the bibliography file.
\bibliographystyle{ACM-Reference-Format}
\bibliography{paper,fernando}

%%
%% If your work has an appendix, this is the place to put it.
%\appendix

\end{document}
\endinput
%%
%% End of file `sample-sigconf.tex'.
